\section{Tabletop Roleplaying Games}\label{sec:ttrpgs}
The following paragraph describes what a Tabletop roleplaying game is and how it is played.
If you are already an experienced player or GM, you can skip this paragraph.
Tabletop roleplaying games are a form of game that is played, as the name suggests, at a table.\\
They require 3 to 8 players, as well as a single person refereeing the game - the \lq Game Master \rq (GM).\\
Each of the players takes on the role of a character, represented by one or more sheets of paper, a so called \lq Character Sheet \rq.\\
The GM describes a situation in a fantastical world to the players, and the players decide how their characters, the so-called \lq Player Characters \rq react.\\
The GM decides, if the actions of a PC would require a die roll.
If they do, the Player rolls a polyhedric die in accordance to the rules of the games and the parameters set by the GM, adds values from their Character Sheet to the roll, and proclaims the resulting number.\\
The GM then decides if the action has failed, succeeded, and by which degree.
As a result, the GM describes how the situation has changed, and the players again describe how they want to react to the evolving situation.
Sometimes, the results of an interaction might make it neccessary to update a character sheet, such as reducing health or adding items to the character's backpack.\\
Many interactions between GM and players happen during a single evening of play, a so-called \lq Session \rq.\\
The result of this simple looping mechanism is a unique, interactive and entertaining story, created and shared by multiple individuals.
Oftentimes, at the end of a session, the players will take their character sheets home (or leave them with the GM), in order to resume the story at the next session.\\
When multiple session share some kind of singularity - typically the cast of characters, or the main storyline - they are called a \lq Campaign \rq.\\
The fantastical world, in which such a story takes place is called a \lq Setting \rq.