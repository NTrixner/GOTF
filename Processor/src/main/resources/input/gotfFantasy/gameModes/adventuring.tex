\chapter{Adventuring}\label{ch:adventuring}
\section{Speed and Travel}\label{sec:timeSpeedTravel}
\subsection{Short-Distance Travel}\label{subsec:shortDistanceTravel}
During Combat or in situations where split-second decisions are important, a character is assumed to be able to tactically move 2m per AP, or run 4m per AP and Stamina spent.
If a character is trying to move stealthy, they are considered to be half as fast during combat or during a time-based encounter.\\

\subsection{Mid-Distance Travel}\label{subsec:midDistanceTravel}
While travelling in areas with a dense layout, like a town, city or a cavern, a character is considered to be able to move 60m per minute.

\subsection{Long-Distance Travel}\label{subsec:longDistanceTravel}
While travelling overland, a character is considered to be able to move 5 km per hour, which means that a well-travelled character should be able to move 20 km per 4-hour interval;
Adventurers are usually considered to be well-travelled.\\

One travelling day is sectioned into six separate 4-hour intervals, three of which are usually spent travelling.

Travelling more than 16 hours reduces a character's maximum Stamina by 1 for each additional hour they spend travelling until they take a rest, at which point one maximum Stamina is restored for each hour rested.\\

For each 4-hour interval, each character in the group can choose a side activity.

These consist of:

\begin{itemize}
	\item Follow Tracks

	\item Ensuring that the group moves stealthy (reduces the group's speed by half)

	\item Lookout for ambushes

	\item Scavenge the area for items

\end{itemize}
This list is not exhaustive.\\

Depending on the area of travel and the group's makeup, there could be more options.
For example, if the group is moving by horse-drawn wagon, one member of the group has to lead the wagon, while others could use the wagon to rest, learn a perk or even craft items.

\subsection{Stealth}\label{subsec:stealth}
If a creature tries to move stealthy in a dungeon, they are considered to be able to move at one-fourth speed, meaning they spend 4 times as many AP for movement, and can only move up to one-fourth their agility in addition to a static action in combat.
If they try not to be noticed while moving through a city, their speed should be treated as half their normal speed.\\
When moving stealthily, a creature makes a stealth check.
If that stealth check is below someone's~\nameref{stat:notice}, they realize something is off - they see moving shadows, hear footsteps, the hiding creature knocks something over etc.
The hiding creature might react to someone becoming aware of them, and the now aware creature might start looking around by making perception checks.

\subsection{Swimming}\label{subsec:swimming}
In still waters, a creature that can swim is usually considered to be able to swim normally.
Swimming for more than 1 hour without break will star to drain a creature's stamina at a rate of 1 per hour.\\
In rougher waters, a creature swimming has to make Agility checks to make sure they can overcome the currents.\\

\subsubsection{Suffocation}\label{subsubsec:suffocation}
A creature starts suffocating the moment they lose their ability to breathe.
A suffocating creature is effectively mute, and loses 1 Stamina per AP of suffocation.
After a suffocating creature has reached 0 Stamina, they start to take 1d6 damage every 5 AP.

\section{Carrying Capacity}\label{sec:carryingCapacity}
Items in the GOTF\_Fantasy are measured relative in the sizes \lq coin sized \rq (CS), \lq tiny \rq (T), \lq small \rq (S), \lq medium \rq (M), \lq large \rq (L), \lq extra large \rq (XL) or \lq extra extra large \rq (XXL).\\
One tiny object is equal to 10 coin sized items.
One small item is equal to 10 tiny items, and so forth.\\
These units are a combination of weight and volume, and containers usually can hold a certain amount of objects.
However, some objects are more or less dense.
For example, while a Backpack might be able to carry the equivalent of one large item in both weight and volume, it might not be able to hold 10 Coins bars -- the straps of the backpack would possibly break.\\
Therefore, GMs and Players should treat these units as larger or smaller depending on the situation if they're dealing with denser or lighter materials.\\
A creature can usually carry one large item per point of STR, but might need a fitting container, like a backpack, to do so.\\

A person is usually able to carry their STR stat in large items.\\

\section{Hazards}\label{sec:hazards}
\subsection{Temperature}\label{subsec:temperature}
In regular environment, characters don't have to fear issues from temperature.
However, very high and very low temperatures can affect a character's health, stamina and can even damage them.\\

In warm and chilly environments, a character's stamina recovery is slowed down to 1 point every 2 AP of resting.
This can be countered by wearing armor with the warming or cooling abilities, respectively.\\

In hot and cold environments, a character takes 1 hot or cold damage per minute.
This damage is applied every 2 minutes, and requires a minimum damage reduction of 2 to be countered.\\

In extremely hot or extremely cold environments, a character takes 1 hot or cold damage per AP.
This damage is applied every 5 AP, and requires a minimum damage reduction of 2 to be countered.\\

\subsection{Falling}\label{subsec:falling}
Falling for 3m deals 1d8 blunt damage on impact.
This damage increases by 1d8 for every additional 3m travelled, to a maximum of 75m, at which terminal speed is reached - meaning maximum damage from falling is 25d8.
A character who can react and move can reduce this falling damage by making an Agility check.
The original Th for this check is 20, and it reduces the falling damage by 1d8.
For every 5 points above 20 that the Agility check reaches, it reduces the falling damage by an additional 1d8.
For example, a character falling 24m would take 8d8 blunt damage on impact.
If said character makes an Agility check with a result of 27, they would only take 6d8 blunt damage instead.

\subsection{Traps}\label{subsec:traps}
Traps are - usually deliberately placed - hazards that consist of a trigger and a -- usually harmful -- effect.\\
When setting a trap, the creator usually rolls a~\nameref{sec:trapHandling} check to set the trap, and a~\nameref{sec:stealth} Check to hide it.\\
When a Character is in the area of a trap, compare the Trap's Finding Th to their Notice.

If the character is actively searching for traps, they gain +5 on their Notice for this comparison.

If the Notice is higher than the Th, the character notices that something is off.

They usually see some sign of the trigger.\\
If the character then decides to actively search for this specific trap, they roll a~\ref{sec:examination} check contested by the Trap's Find Th.
If they find the trap, they can then attempt to disable it by making a~\nameref{sec:trapHandling} check contested by the Trap's Trap Handling Th.\\
If the disarm attempt succeeds, the trap is disarmed.

If it fails, the character currently misses the knowledge or expertise to disarm this trap.\\
If the disarm attempt fails by 5 or more, the trap is sprung, and the effect is triggered immediately, regardless of if any characters are in the effect's area.\\
Additionally, if a character skilled in Trap Handling beats the Th of a Trap by 5 or more, they might be able to gather some or even all the materials used in the creation of the trap.\\
If a character triggers a trap, they usually realize that they have triggered something - they feel a tripwire that is being taught and a click coming from the wall, or they realize that the tile they stepped on is giving away, or something similar.

They then have the chance to react, gaining an appropriate check against the traps' effect if they react correctly.\\

\section{Senses}\label{sec:senses}
Any creature in the game possesses a certain set of senses, that they can use to perceive their surroundings.
For most creatures, one of these senses is their primary sense.
A creature that loses their primary sense for some reason, or cannot rely on it for a given task, should treat their Perception to be halved.
Unless otherwise specified, creatures have the senses of Sight, Smell, Hearing, Touch and Taste.
Additionally, unless otherwise specified, a creature's default sense is Sight.

\subsection{Cover}\label{subsec:cover}
For the purposes of perception, an object or creature is considered to be behind cover if the form of cover blocks the perceiving creature's primary sense in some form.
For example, a creature behind a pillar would be considered to be behind cover for the purpose of Sight, but not for the purpose of Hearing.
A creature that is not completely behind cover, but partially (such as by a belly-high wall or by shadow), is considered to be Partially Covered.

Any perception checks and attacks against a creature or object that is Partially Covered suffer a -4 Circumstance Malus.
Additionally, a creature that is physically Partially Covered takes half damage from any effects that are partially blocked (an explosion, for example), unless their cover is completely destroyed by the effect.

Any perception checks and attacks against a creature or object that is Covered suffer a -10 Circumstance Malus.
Additionally, a creature that is physically Covered takes no damage from any effects that are blocked (an explosion, for example), unless their cover is completely destroyed by the effect.

\subsection{Non-standard forms of perception}\label{subsec:otherSenses}
This is a list of forms of perception that are not considered standard:
\begin{itemize}
	\item Nightvision
	A creature with Nightvision can see in low light as if it was daylight, and in darkness as if it was low light.
	This functions to a certain range that is specified with the sense.
	In complete darkness, such a creature is still blind.

	\item Echolocation
	A creature with Echolocation can build a mental image of their surroundings by sending out sonar waves, and capturing them as they are reflected by solid objects.

	\item Magnetoception
	A creature with Magnetoception can perceive the magnetic fields of the planet.
	Such a creature can easily identify their own heading.

	\item Electroreception
	A creature with Electroreception is the ability to feel the electric fields surrounding them, sometimes creating such fields themselves.
	As a result, such creatures can feel changes in these fields, such as obstacles or creatures passing through them.

	\item Heat Vision
	Heat Vision allows a creature to perceive the infrared light that is emitted by objects that have a heat signature.
	Heat Vision works through thin objects, such as sheets or wooden planks.

	\item Arcane Vision
	A creature with Arcane Vision can perceive the magical energies in their surroundings.
	Most creatures, as well as magical items, plants, buildings, terrain and other objects have a form of magical signature that can be used to identify one's surroundings.
	This sense has a certain range, but functions around and through objects.
	As such, Arcane Vision is similar to Electroreception, but using magical fields instead of electrical ones.
\end{itemize}