\chapter{Social Interactions}\label{ch:socialInteractions}
In order to adjudicate a social encounter, a GM should track the opposing side's resolve against any given argument, their annoyance, and the group's overall progress in convincing the opposing side of their position. \\
Depending on the difficulty of the encounter, these numbers range in the following areas:\\
Resolve: 1--24\\
Annoyance: 0--5\\
Progress: 0--5\\
These values start at a value that the GM decides, but annoyance and progress usually start at 0, unless the characters have interacted with the opposing side previously, or if the opposing side has already heard of the characters.\\
Resolve is basically the bonus that is added to any given Th of a check that the characters have to roll against the opposing side.\\
Whenever a PC makes an argument, the GM decides if that argument can change the opposing side's mind and if it can fail to do so.\\
If the argument can both succeed or fail, the Character rolls the given check against the NPCs Resolve + a specific value that the GM decides based on the PCs argument and the NPCs annoyance.\\
If the argument can not fail, the GM adds one point to the Progress.\\
If the argument can not succeed, the GM adds one point to the Annoyance.\\
If the annoyance or progress reach 5, the social interaction ends.
If it ends because the group annoyed the opposing side too much, they failed the encounter.
If they brought forth enough good arguments to reach a progress of 5, the party wins the argument.\\
Of course, these values can be changed based on the situation.