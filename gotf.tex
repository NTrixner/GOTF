\documentclass{article}
\textwidth=15cm
\oddsidemargin=0.5cm
\setlength{\parindent}{0pt}
\setlength{\parskip}{1pt}
\usepackage[margin=3cm]{geometry}
\usepackage[toc,page]{appendix}
\usepackage[table]{xcolor}
\usepackage{makecell, multicol, wrapfig, longtable}
\usepackage{hyperref}
\usepackage{pgf}
\usepackage{siunitx}
\usepackage{xfrac}
\usepackage{textcomp}
\usepackage{amsmath}
\hypersetup{
	colorlinks,
	citecolor=black,
	filecolor=black,
	linkcolor=black,
	urlcolor=black
}

\setcounter{secnumdepth}{0}
\setcounter{tocdepth}{1}

\newcommand{\passus} [1] {%
  \SI{#1}ps (\pgfmathparse{1.5*#1}\pgfmathprintnumber[precision=2]\pgfmathresult~m, \pgfmathparse{5*#1}\pgfmathprintnumber[precision=2]\pgfmathresult~ft)%
}

\newcommand{\leg} [1] {%
  \SI{#1}leg (\pgfmathparse{1.2*#1}\pgfmathprintnumber[precision=2]\pgfmathresult~km, \pgfmathparse{0.746*#1}\pgfmathprintnumber[precision=2]\pgfmathresult~mi)%
}

\newcommand{\fin} [1] {%
  \SI{#1}fin (\pgfmathparse{1.5*#1}\pgfmathprintnumber[precision=2]\pgfmathresult~cm, \pgfmathparse{0.6*#1}\pgfmathprintnumber[precision=2]\pgfmathresult~")%
}

\newcommand{\pugnus} [1] {%
  \SI{#1}pugnus (\pgfmathparse{0.33*#1}\pgfmathprintnumber[precision=2]\pgfmathresult~kg, \pgfmathparse{10*#1}\pgfmathprintnumber[precision=2]\pgfmathresult~fl oz)%
}

\newcommand{\stone} [1] {%
  \SI{#1}stone (\pgfmathparse{6*#1}\pgfmathprintnumber[precision=2]\pgfmathresult~kg, \pgfmathparse{13*#1}\pgfmathprintnumber[precision=2]\pgfmathresult~lb)%
}



\begin{document}
	\title{GOTF \\ Generic Open Tabletop Framework}
	\
	\author{Nikolaus Trixner}
		
	\maketitle
	
	\tableofcontents

\part{Introduction}\label{part:intrudction}
\section{Tabletop Roleplaying Games}\label{sec:ttrpgs}
	The following paragraph describes what a Tabletop roleplaying game is and how it is played.
	If you are already an experienced player or GM, you can skip this paragraph.
	Tabletop roleplaying games are a form of game that is played, as the name suggests, at a table.\\
	They require 3 to 8 players, as well as a single person refereeing the game - the "Game Master" (GM).\\
	Each of the players takes on the role of a character, represented by one or more sheets of paper, a so called "Character Sheet".\\
	The GM describes a situation in a fantastical world to the players, and the players decide how their characters, the so-called "Player Characters" react.\\
	The GM decides, if the actions of a PC would require a die roll.
	If they do, the Player rolls a polyhedric die in accordance to the rules of the games and the parameters set by the GM, adds values from their Character Sheet to the roll, and proclaims the resulting number.\\
	The GM then decides if the action has failed, succeeded, and by which degree.
	As a result, the GM describes how the situation has changed, and the players again describe how they want to react to the evolving situation.
	Sometimes, the results of an interaction might make it neccessary to update a character sheet, such as reducing health or adding items to the character's backpack.\\
	Many interactions between GM and players happen during a single evening of play, a so-called "Session".\\
	The result of this simple looping mechanism is a unique, interactive and entertaining story, created and shared by multiple individuals.
	Oftentimes, at the end of a session, the players will take their character sheets home (or leave them with the GM), in order to resume the story at the next session.\\
	When multiple session share some kind of singularity - typically the cast of characters, or the main storyline - they are called a "Campaign".\\
	The fantastical world, in which such a story takes place is called a "Setting".

\section{What the GOTF is}\label{sec:gotfexplanation}
	For many, following a predefined ruleset is the easiest solution when it comes to TTRPGs.
	There have been many more or less successful systems in the past, and there is something for most flavors.
	It is not the target of the GOTF to replace these systems, and we don't espouse to be an "end all be all" system for every single player out there - that would be arrogant.\\
	Instead, the target of the GOTF is to take a setting-gnostic system (called RLP, the Ragged Lands Pen and Paper System), which was designed by the author of this document, and make it more generalized, so it can fit many different worlds, flavors and aesthetics.\\
	The GOTF is a meta-system designed to make it easy to create, modify and share RPG Rules Systems.\\
	It is based around a single Core Mechanic - the roll of a D20 plus modifiers, but adds multiple systems on top of that, while being completely modular.\\
	The basics of a GOTF system are: Game Modes, Attributes, Perks, Origins, and Inventory.
	We will go into more detail regarding these entities in the following chapters.\\
	This document is not designed for beginners, and not for people who just want a simple RPG System they can pick up and try out.
	For that, try the file "GOTF\_Fantasy.pdf", which is a GOTF-based rules system designed for simple fantasy adventures - think of your generic humans, elves, dwarves etc.

\section{How to use this document}\label{sec:howtouse}
	In the following chapters, this document explains how to create a GOTF-based system.
	It first describes the base mechanic, as well as how to adjudicate actions, handle rounding, modifier types, etc.
	It then goes over the base components of a GOTF-based system - Game Modes, Attributes, Perks, Origins and Inventory.
	To a large extent, most of these systems do not provide fixed numbers or strong guidelines.
	The reason for that is that the GOTF is designed to work with different levels of power-fantasy.
	The balancing of numbers is left to the GM.
	Note, that this document is licensed under the WTFPL.
	This means that this document has no legal restrictions to any use, both commercial or private.
	You can also distribute this document freely and without charge.
	Unlike with other systems and licenses, this will never change.

\part{Basic Rules}\label{part:basicRules}
	\section{The Base mechanic}\label{sec:basemechanic}

\subsection{The Meta Mechanic}\label{subsec:metamechanic}
At the beginning of a scene, the GM describes the scene, as well as any points of interest.
The Players then decide what their player characters want to do.
These actions are then adjudicated.
After actions are adjudicated, the scene has changed and the GM describes the changed scene, or the next scene.

\subsubsection{Action Adjudication}\label{subsubsec:actionadjudication}
Whenever an action happens, it is adjudicated.
The GM decides two things: A) Can the action succeed.
If the action is impossible to succeed, the GM should usually tell the Player.
B) Can the action fail.
If an action can not fail, there is no need to make a roll.\\
If an action is both possible to fail as well as possible to succeed, a character must make an appropriate check.
Most of these checks are Attribute Checks, as described in the next section.\\


\subsection{Attribute Checks}\label{subsec:attributechecks}
Every check is solved by a roll of a D20 You add a specific modifier to your roll, and maybe get additional bonuses if you're trained in that specific task.
Whenever you roll a check, you compare the outcome of your roll to a Threshold (Th).
If your roll beats that Th, you are able to perform that task.
If your value is lower than the Th, you can not perform the task.
If your roll value and the Th are equal, you roll again.

\subsubsection{Contests}\label{subsubsec:contests}
Sometimes, two creatures compete in a task.
In order to get the outcome, both people roll a D20 and add their relative modifiers, and optional additional dice.
The one with the higher result wins the contest.
If both rolls are the same, they are rolled again until they are not the same anymore.

\subsubsection{Helping Others and working together}\label{subsubsec:helping}
Sometimes, a character may be able to help another character with a difficult task.
In such a case, the player describes how exactly they are helping, and the GM adjudicates if it's one of three general scenarios.\\
\textbf{Guided acting} occurs if one character tells another character what to do in order to achieve a task.
In this case, the acting character may add a bonus set by the System to the roll for every rank of Proficiency Perk that the guiding character has for said task.\\
\textbf{Direct Help} If a character helps another character directly in a certain task, the helping character may have to roll for the specific way in which they are helping.
If they succeed, the Threshold of the original task is reduced.\\
\textbf{Working together} If two or more characters are working together, every character makes a roll.
Depending on the nature of the task, either the group succeeds if one of them succeeds, or the group fails if one of them fails.\\
Depending on the task, working together might change the Th of the given task.\\

\subsubsection{Retries}\label{subsubsec:retries}
Rolls determine ability, not (just) luck.
his means, that unless the circumstances change, the outcome of the roll is fixed.
Retries are not an option.
If the circumstances change, rolls can be retried.
This can mean stress, new knowledge, new abilities or a changed environment.


\subsection{Rounding}\label{subsec:rounding}
Every time a division happens and the result would be a fraction when an integer is needed, you should round up.

\subsection{Modifiers Types}\label{subsec:modifiertypes}
When modifiers are applied to bonuses and maluses, they usually have a type associated with them.
In general, a creature only profits the strongest bonus of a given type, and only suffers the strongest malus of a given type, but different types of bonuses and maluses are summed up.
Additionally, some bonuses and maluses apply dice instead of flat values.\\
In such cases, the largest amounts of the same sized dice (both positive and negative) apply to the checks.
Additionally, positive and negative dice of equal size cancel each other out (for example, a creature with +1d4 and -1d4 has a bonus of 0) before the actual roll.\\
Lastly, some modifiers are multiplicative.
Multiple multiplicative modifiers of different types are treated as if they were themselves summations, meaning the total of a xN and a xM multiplier is x(N+M-1) - basically, a multiplicative Modifier is a \lq +X\% \rq bonus.
Multipliers are applied first, fractions second, and then dice bonuses/maluses and flat bonuses/maluses.\\

\subsection{Passage of Time}\label{subsec:passageoftime}
Actions usually take at least some time.
A GM will adjudicate an action at the moment that it starts, and then time will pass while they are acting.\\
The amount of time passing should be clear to the players, and the GM should always be aware what character is performing which action.\\
The amount of time passing during an action is determined by the action itself, and varies during different frames of gameplay.
For example, overland travel might be tracked in hours or even days, while searching a room might take minutes.
All the while, combat will only take a few minutes, maybe even just seconds.

\subsubsection{Action Points}\label{subsubsec:actionpoints}
Action Points are a special unit used within the rest of the document.
Action Points (\textbf{AP}) are used to measure time in action-heavy scenes.
The real duration of an Action Point is determined by the system.

\part{Creating a system}\label{part:systemCreation}
	\section{Game Modes}\label{sec:gamemodes}
GOTF groups multiple interactions into so-called scenes.
A scene happens whenever the players' goals are at risk in some form.
Not every group of interactions is automatically a scene - only situations in which a \lq Dramatic Question \rq is happening can be counted as a scene.
In such cases, how the player characters resolve the situation decides the Game Mode they're operating in.
In a GOTF-based system, every character has a level for each Game Mode that is designed into the system.
For example, the GOTF\_Fantasy has the modes Combat, Social Interaction and Adventuring.
When designing a Game Mode, the GM should keep the following questions in mind:
\begin{itemize}
    \item What distinguishes this Game Mode from the other Game Modes
    \item What actions are typical for the characters to perform in a specific Game Mode
    \item What measurements are going to be typical in a Game Mode
    \item What resources are used in the Game Mode
\end{itemize}

A character performing a trained action in a specific Game Mode (See \lq ~\nameref{sec:perks} \rq below) can add their relevant Level to relevant checks.\\
When a party successfully prevails through a scene in a Game Mode, the characters gain 1 Experience Point (XP) in that Game Mode.\\
Some Scenes are more important or dramatic to the story than others.
These Scenes may provide 2 XP in that Game Mode, or in extreme cases even 3 XP.\\
A character immediately levels up their specific Game Mode when they have gained a certain amount of XP for that Game Mode.
By default, the formula for the required XP for a given level is:\\
$XP_{l+1} = XP_{l} + l * 4$\\
Where $XP_{1} = 0$ and $XP_{2} = 1$\\
This means that a character starts at level 1 in each Game Mode, and gains level 2 after one Scene in that Game Mode, and level 3 after 5 Scenes, and so forth.\\

By default, each time a character gains XP in a Game Mode, they gain Learning Points (LP).\\
Learning Points are used to acquire new Perks (See \lq ~\nameref{sec:perks} \rq below).\\
The amount of LP a character gains when gaining XP equals their current level times 100.\\
If a character gains more than one XP in a level, they gain LP as if the would have gained each XP after the other - meaning that a character with 4 XP who gains 3 XP would gain 200 LP for the first XP, and 600 for the second two.\\

\section{Attributes}\label{ch:attributes}
The GOTF uses Attributes to describe a character's abilities in a numeric way.
There are three types of attributes: Active, Defensive and Pools.\\

\subsection{Active}\label{subsec:activeattributes}
Active Attributes are static values that describe a character's physical and mental abilities.
In the GOTF\_Fantasy, these include Strength, Vitality, Dexterity, Agility, Intellect, Perception, and Charisma.\\
Active Attributes range between 1 and 12, and can not leave this range without special reasoning.
A system can decide how a player allots these values at character creation, but suggestions are rolling and then assigning, point distribution with a minimum value, rolling directly, and predefined value lists.\\

\subsection{Defensive}\label{subsec:defensiveattributes}
Defensive Attributes are attributes that can be targeted by some type of effect.\\
In the GOTF\_Fantasy, these include Dodge, Notice and Willpower.\\
Defensive Attributes are usually derived from Active Attributes.\\
Each Defensive Attribute has three values: The modifier, the Lower and the Higher value.\\
The Modifier is usually equal to the active attribute that it derives from, adding Game Mode levels if the character is trained in the defensive attribute, or gains proficiency in the attribute in some other manner.
The Lower Value equals 8 plus the Modifier, and the Higher Value equals 21 plus the Modifier.\\
Whenever a check targets a creature's defensive attribute, the result of that check is compared to the Lower and Higher Value first.
If the check is below the Lower Value, it fails.\\
If it is higher than the Higher Value, it succeeds.\\
If it is between (including the Lower and Higher value itself), the target creature can make a d20 roll and add their Modifier to it.
If that roll is higher than the original check, the negative effects of the original check is reduced - sometimes by half, sometimes completely, as designated by the action itself.\\

\subsection{Pools}\label{subsec:poolattributes}
Pools are a form of depleting resource that a creature has.\\
In the GOTF\_Fantasy, these are Health, Mana and Stamina.\\
A Pool Value has a maximum value, a current value (that is smaller than or equal to the maximum value) and the current Value can be reduced by actions of the creature itself or by other effects.
When the current Value of a Pool reaches 0, some negative effect afflicts the creature.
A system may define sub-ranges of a pool that already have some negative effects, in order to promote resource management.\\
Each Pool Value should also have a way of regenerating it.\\
Similar to Defensive Attributes, Pools can also be based on Active Attributes.
If that is the case, any perk that increases a Pool would also add the Active Attribute value to the Pool's Maximum Value, and any change of the Active Attribute would trigger a recalculation of the Pool.\\
Determining the starting Value of Pools is up to the system, but the Origins of a Character are a typical source.\\

\section{Origins}\label{sec:origins}
Origins describe the character in amore fundamental, and historic way.
In the GOTF\_Fantasy, this includes Species, Culture and Background.\\
A system can have as many or few types of Origins as desired, but somewhere between 2 and 5 is easier to track.\\
The boni from these Origins can be manyfold, and the designer of a system is completely unlimited, but common suggestions would be:
\begin{itemize}
    \item Gaining +2, +1, -1 or -2 on an Active Attribute.
    This might also increase or decrease the maximum value of that Active Attribute.
    \item Gaining a certain type of modifier in specific tasks, typically a +1d4 or a +1d6.
    \item Gaining temporary bonuses that can be used once, and then not anymore for a significant amount of time.
    \item Static bonuses, such as certain types of perception, movement or communication.
    \item Bonuses that apply when interacting with others of the same Origin, or when interacting with the Origin itself.
\end{itemize}

\section{Perks}\label{sec:perks}
Perks are modular forms of powers and abilities that a character has learned.\\
These can be spells, increases for active or pool attributes, special movements, superpowers that the character becomes more accustomed to, new software on a mech, knowledge a character has gained, the limits are only decided by the system.\\
In general, Perks can be described in the following way:\\
\begin{itemize}
    \item Name
    The name of the Perk
    \item Requirements
    This describes what requirements a perk has.
    This can include other perks, a certain origin, a certain Game Mode Level, or in-game feats that aren't described in the system itself.
    \item Costs
    This describes the costs of a perk in LP.
    A character who learns a perk has to spend this amount of LP.
    Sometimes, a system will also denote how much time is required to learn a perk.
    In such a case, it is a good idea to define a relationship between LP and amount of required time, since \lq costs more LP \rq and \lq takes more time to learn\rq are both synonyms for \lq is more difficult to learn \rq.
    \item Tags
    The tags, or descriptors of a perk.
    These might include if the perk can be trained in multiple ranks, if it is a \lq passive \rq (aka always active) perk or an \lq active \rq (aka some type of action the character gains) perk, if it is a spell, if requires memorizing, etc. etc.
    This might also include how and where a character can learn a certain perk - books, scrolls, trainers, the internet, if it costs money to learn something, etc.
    \item Description
    This includes a description of the in-game explanation of the perk, as well as a description of the effect that the perk has in the terms of the system rules.
    In case of perks that have multiple ranks, this also includes the level progression (How many LP a higher Rank of the perk costs, as well as the effect that it has on higher levels)
\end{itemize}

\subsection{Group Perks}\label{subsec:groupperks}
Some perks are organized in groups.\\
In the GOTF\_Fantasy this includes Armor Proficiencies, Weapon Proficiencies, Skills, Spells and Martial Perks.\\
Organizing Perks into groups and giving them group mechanics can be used to increase immersion, give thematic cohesion, and curb hyper-specialization.\\
The latter can be done by calculating the Perk Rank costs of the Perks in a Group by the total amount of Ranks of Perks in that group.\\
As an example, the GOTF\_Fantasy uses a Group Rank Progression for Skills.
This means that every new Skill and every higher Rank in an already learned skill are equally difficult to learn, resulting in there being a distribution of skills amongst the PCs, and not one character having \lq all the skills \rq, while the other characters focus on other issues.\\
Other ways of grouping perks include common mechanics, common resources, similar effects but for different areas of the game (like all proficiency and skill perks in the GOTF\_Fantasy) and all perks in a group using the same Pool.\\

\section{Inventory}\label{sec:inventory}
While the GOTF does not necessitate that any given system uses an itemization system and track equipment or money, many traditional tabletop RPGs have such a system, and so here are some basic guidelines to create a working item system:\\
Any character can carry a certain amount of items, possibly based on a type of container they are wearing.
Each item comes in a different size category, and the different size categories use a factor of 10 with each other.\\
For example, the GOTF\_Fantasy has coin sized (CS), tiny (T), small (S), medium (M), large (L) and extra-large (XL) items.
One tiny item is the same size as 10 coin sized items.\\
A character can generally carry an amount of large items equal to their strength.\\
\\
The value of items is defined by some form of currency, but in general a baseline for the value of such a currency is needed.\\
A good base value could be LP - in such a case, a designer could create, for example, a 1:1 ratio between LP and money.
While it wouldn't be possible to convert money into LP or vice-versa (other than perhaps at character creation), the power of items could then be determined alongside perks.\\
This comparison wouldn't work for temporary or consumable items though, since gaining a new ability would be exponentially more valuable than a limited-use object.\\
In such a case, a dividend of 5 is suggested, but could also be 2 or 10, or something else.\\


\part{License}\label{part:license}
	Copyright © 2000 Nikolaus Trixner\\
	This work is free.
	You can redistribute it and/or modify> it under the  terms of the Do What The Fuck You Want To Public License, Version 2, as published by Sam Hocevar.
\end{document}