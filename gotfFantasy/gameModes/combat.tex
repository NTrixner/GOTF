\chapter{Combat}\label{ch:combat}
Whenever a combat situation erupts, the GM starts the battle by deciding if one or multiple participants are surprised.
These combat participants are not allowed to act until the others have acted.
Then, the GM lets the characters act in a specific order that they decide is the most fitting.

Every action in combat takes a specific amount of Action Points.
Action Points are a measure of time, with one Minute consisting of about 30 Action Points.
After an action was resolved, the acting character has to wait for the amount of Action points his action took before being able to act again.
If no participant is acting, the GM counts down Action Points until a participant is again able to act.

As an alternative approach, an experienced GM may decide to not use AP at all.
This can make the game more flowing and immersive, but it may prove to be difficult to balance.


\section{Actions in combat}\label{sec:combatActions}

\subsection{Combining Actions}\label{subsec:multipleCombatActions}

Broadly speaking, there are two types of behaviour during combat -- movement and actions.
Whenever a character acts, they can combine one movement and one action.
After acting, a character cannot act until time has passed equal to the AP they used.
This means, that the combatant with the least amount of total AP spent can act next.

\subsection{Combat Actions}\label{subsec:combatActions}

\subsubsection{Offensive}\label{subsubsec:offensiveCombatActions}
\begin{itemize}
\item Attack [AP depending on weapon action]

In order to attack an opponent, the character has to be in reach of said opponent.
This reach is defined by the character's weapon.
He then has to make an attack roll:\\
\\
Attack roll = 1D20 + STR/DEX/AGI (depending on the weapon) + [Combat Level + XD4] + misc
\\
\\
Magic attacks (such as runes and spells) use AGI for melee attacks, and DEX for ranged attacks.
The D4s and level can only be added if the character is trained to a certain amount in the attack they are making.
If the attacker has an attack roll that is lower than the defender's Dodge, the attack misses.
If the defender is unaware of the attack, they can not react.
Otherwise, the defender can react in one of three ways: Dodge, Block or take the hit.
All of these actions - except taking the hit - require a successful check and a specific amount of Stamina.
The amount of stamina used for a defensive action is based on the amount of  AP that the attack costs.
The Th for the defensive check is the attack roll.
The defensive action may prevent any damage from happening.
Otherwise, the attacker rolls damage according to his weapon.
The weapon entry states what attribute is added to the damage roll, and how much of it is added.
When calculating this value, remember to round up.
The defensive action and the defender's armor may reduce the resulting damage, and the rest is dealt to the defender's health.
If the attack deals multiple types of damage, each type of damage is treated separately.


\textbf{Fumbles and Critical hits}
When you roll a Natural 20 on a Weapon attack roll, you automatically hit as if the defender would take the "Take Hit" action and deal additional damage.
This means that any weapon damage die is rolled twice, taking the total result when determining the damage of the attack.
This is called a "critical hit", and some enemies are immune to it or have a chance to resist a critical hit.
When you roll a Natural 1 on a Weapon attack, you automatically miss your attack.

\item Trip [AP depending on weapon; 10 Stamina]

Make a melee attack roll against a standing opponent, using additional Stamina.
Your attack roll is contested by the enemy's Dodge or Block Action.
This defensive action does not cost any Stamina.
The enemy gains a Stability bonus to their roll, equal to the amount of feet they have.
When your enemy fails, they drop prone.\\
Unlike a normal attack, a trip does not deal damage.\\

\item Disarm [AP depending on weapon; 10 Stamina]

You attack one opponent that is armed with a manufactured weapon.
Make a melee weapon attack roll, taking an additional 5 Stamina, opposed by the enemy's weapon attack roll.
If you succeed, the enemy drops their weapon to the floor.
You don't deal damage with this attack.\\

\item Wrestle [AP depending on action]

    Wrestling is a form of unarmed combat that doesn't involve hitting or punching.
    Instead, in includes grabbing, pinning, throwing or pushing enemies.
    These actions include a check called "Wrestle checks".
    A wrestle check is a STR check by the actor, contested by a STR or AGI check by the non-actor.\\
    Oftentimes, the difference between these checks determines the magnitude of the outcome.
    Additionally, the size of a creature determines a bonus or malus to a wrestle check.\\
    If the non-acting creature is larger than the actor, it gains a size bonus equal to +4 per size difference to the check.\\
    If the non-acting creature is smaller than the actor, it suffers a size malus equal to -4 per size difference to the check.\\
    Additionally, if a creature that you would move (via the throw or push action) is wearing armor, they gain a weight bonus to the check.\\
    A creature wearing light armor gains +2, a creature wearing medium armor gains +4 and a creature wearing heavy armor gains +8.

    \begin{itemize}
    \item Grab [3 AP]

        You attack a creature as if you were making an Unarmed Strike attack.\\
        When the attack hits, make a Wrestle Check.
        If you succeed, the creature is grabbed by you.
        You can grab as many creatures as you have appendages, divided by 2.\\
        If a target creature is larger than you, you need twice the appendages to grab them per size category difference.
        For example, a medium creature grabbing a large creature would need 4 appendages.
        A medium creature grabbing a huge creature would need 8 appendages.\\
        If a target creature is smaller than you, you only need one appendage to grab it.\\
        If you use half or more of your pedal appendages (legs) to grab creatures, you fall prone.

    \item Escape [3 AP; 5 Stamina]

        You try to escape from a creature that has grabbed or pinned you.\\
        To do so, make a wrestle check.\\
        If you are pinned, you suffer a -10 situational malus to this check.
        If you succeed while pinned, you are no longer pinned and restrained, but still grabbed and prone.\\
        If you succeed while grabbed, you are no longer grabbed.\\
        If you succeed by 5 or more, you can decide to grab the other creature instead (no further checks required).

    \item Pin [4 AP; 10 Stamina]

        You try to pin a creature that you have grabbed and push them to the ground under you.
        Make a Wrestle check.\\
        If you succeed, the creature falls prone and is restrained.\\
        If the target creature succeeds by 5 or more, it is no longer grabbed by you.

    \item Throw [4 AP; 10 Stamina]

        You try to throw a creature that is grabbed by you.
        To do so, make a Wrestle check.\\
        If you succeed, the creature is thrown a distance equal to 1m for each point of difference between the checks.\\
        For every 2m that the creature is thrown before they hit the ground or a wall, it takes 1d6 blunt damage.
        If the creature takes at least one damage, it falls prone where it lands.\\
        If the creature succeeds on the wrestle check by 5 or more, it is no longer grabbed by you.

    \item Push [AP depends; Stamina depends]

        As part of a movement, you move a creature that is grabbed by you.
        To do so, make a wrestle check.\\
        If you succeed, you move the creature up to a distance equal to your movement distance, or the difference between the checks, whichever is higher.
        You also spend 2 Stamina per meter moved this way.

\end{itemize}

\item Counter [3 Stamina per counter attack AP]

When a defensive action (except taking the hit) reduces a melee attack's damage to 0, or if an attacker rolls a natural 1 on a melee attack while they are in reach for a melee attack by the defender, the defender can use the opportunity given by the botched attack to counter it with their own melee attack.
Doing so is quite strenuous, requiring Stamina instead of the AP cost of the counter-attack.


\item Fight defensively [1 AP]

A character can decide to fight defensively for the duration of one Action Point.

If they do so, they gain +4 on rolls for defensive actions and +2 on Dodge for that duration.
A character may declare to fight defensively for a specific amount of time, or until a certain condition is met.
In the latter case, they act again on the AP count after that condition triggers.

\subsubsection{Defensive}\label{subsubsec:defensiveStaticCombatActions}


\item Dodge [Defensive; Stamina depends on Armor worn]

Dodging requires a Dodge chack against the attack roll.
If the defender succeeds this check, they move 1.5m to a free spot and take no damage.
If there is no free spot, or the defender's roll failed, they take normal damage instead.
Dodging can be used against melee and ranged attacks.
See \ref{stat:dodge} for more information.
The Stamina spent on this defensive action is based on the type of armor that the defender is wearing.\\
If they are wearing no armor or cloth armor, it costs 4 Stamina.\\
If they are wearing light armor, it costs 8 Stamina.\\
If they are wearing medium armor, it costs 12 Stamina.\\
If they are wearing heavy armor, it costs 16 Stamina.

\item Block [Defensive; Stamina depends on weapon]

Blocking requires a Weapon defense roll against the attacker's attack roll.
If the defender succeeds this check, they roll a weapon defense damage roll and add the result to their armor rating before reducing the attacker's damage by the total.
(When wielding two weapons or a weapon and a shield, the defender may choose one of the two).
Blocking can only be used against melee attacks unless otherwise stated by the blocking weapon.

\item Take Hit [Defensive; 0 Stamina]

Taking the hit is the default defensive action, therefore it doesn't cost anything.
The attacker automatically hits, and the weapon damage is reduced by the defender's Armor Rating.
Taking the hit can be used against melee and ranged attacks.


\subsection{Movements}\label{subsec:combatMovements}
The following actions include types of movement.
Some of them refer to a value called the "movement distance" of a character.
This value is calculated like this:\\
\\
Movement Distance = 3 + (AGI + [Combat Level + Misc]) / 3
\\
This formula results in the following table.\\

\rowcolors{2}{lightgray}{white}
\begin{tabular}{r r}
    AGI [+ Level + Misc] & Movement Distance\\ \hline
    1 - 3 & 4m \\
    4 - 5 & 5m \\
    6 - 8 & 6m \\
    9 - 11 & 7m \\
    12 - 14 & 8m \\
    15 - 17 & 9m \\
    18 - 20 & 10m \\
    21 - 23 & 11m \\
    24 - 26 & 12m \\
    27 - 29 & 13m \\
    etc. & etc. \\
\end{tabular}\\~\\

The Combat Level and D4s are only included if the character is trained in the ~\nameref{perk:mobile} perk.\\

\subsubsection{Grids}

While the GOTF\_F does not necessitate a grid for combat, a GM might include a grid to enhance tactical combat.
When doing so, it is a good idea to use a square grid, where a square represents 1 square meter.\\
Diagonal movement on a grid can then be handled in this way:\\
Count every second diagonal movement as if it were 2 meters.
Do so, even if the diagonal movements are disjointed.\\
For example, a character might move 1 diagonal, 2 straight, 1 diagonal, 1 straight and 1 diagonal again.
This would count as 7 meters in total, since the second diagonal counts as two meters.

\item Walk [4 AP]

You move in a strategic manner up to your movement distance in meters.

\item Run [4 AP, 8 Stamina]

In order to run to a specific position, a character has to spend stamina.
As a result, the character can move up to twice their movement distance in meters.

\item Step [0 AP]

A character can move 1m without paying any amount of AP to do so, as long as they perform a regular action as well.

\item Draw/Sheathe Weapon [AP cost depends]

When not armed, a character first has to draw their weapon.
Also, if they want to change weapons, they usually have to either drop the weapon they're wielding (which doesn't cost any AP) or sheathe it, before drawing a different weapon.
The AP to do so is specified by the weapon.
A character can draw or sheathe their weapon as part of a different movement, only paying the higher amount of AP.
For example, drawing a certain weapon might cost only 2 AP.
A character drawing said weapon as part of a walk would pay 4 AP.
If a weapon would cost 8 AP to draw, the character could combine a walk and draw, and still only pay the 8 AP for the draw.
\end{itemize}