\chapter{Coins}\label{ch:coins}
The value of items, as well as the costs of crafting recipes, and the cost of some perks, is measured in Base Coins, usually shortened to just coins.
Coins are a representation of a base coin that a currency might have.
In the game, a culture, nation or continent will have its own form of currency, and a Base Coin will represent the lowest type of coin in this currency.
It is highly recommended to use or create a currency system for a game, since coins are CS-sized items, and therefore require inventory space.\\
Additionally, having different currencies can create immersion.
For example, a party might have to change their money when they travel to a new nation, or to a new continent.
Also, they might find ancient coins of a long forgotten empire, one that might have to be exchanged, or can only be sold to collectors.

\section{Example: Midlands Culture Currency}\label{sec:coinExample}
One example for a easy to use and fairly standard Coinage system is the Currency of Midlands culture.
The midlands culture has Copper Coins, Silver Coins, Gold Coins and Platinum Coins.\\
A Silver Coin is worth 100 Copper Coins, a Gold Coin is worth 100 Silver Coins, and a Platinum Coin is worth 100 Gold Coins.
Whenever a price in this document is listed, assume that it is listed in "Copper Coins".

\section{Exchanging Rates}\label{sec:exchangeRates}
Exchange rates between currencies vary widely and from town to town, exchange office to exchange office.
First, most smaller towns, or even larger cities, don't have exchange offices at all.
One would find such places mostly in harbor towns or towns that are close to borders, as well as capital cities.\\
As a general rule, use Base Coins as the basic value.
A richer country's currency will be up to 30\% more valuable than that of a poor country.
As such, a party might get only 70 Base Coins' worth of currency when they exchange 100 Base Coins of the less valuable currency.\\
An exchange office will also charge between 10\% and 50\% in exchange costs.
This price will vary on multiple bases, one of which is the currencies that are being exchanged.
A country that is allied with the origin of the exchanged currency, and does trade with them often, will most likely have relatively cheap exchange prices.
A country that is at war with the origin of the exchanged currency will most likely charge a lot for exchanging currencies.
A country that has no contact with the origin of the exchanged currency might be unable to exchange it at all.
Another factor for the exchange costs is the exchange officer themselves.\\
Some exchange officers are just greedy and charge a higher price than is required.
Some might even skim off the top of the exchange costs, a crime in many countries.

\section{Trading and Bartering}\label{sec:trading}
When a player wants to trade items, they have to find a vendor first.
Not every vendor sells or buys every type of item.\\
Not every vendor buys items, some only sell them, and the other way around.
Not every vendor can be bartered with, depending on their personality, culture and local laws.\\
The general amount of money that a vendor can use and the stock that they carry is dependant on the success of their shop, the general economy of their surroundings, as well as their renown.
For example, a small blacksmith in a quiet backwater village will usually have up to 500 Coins to trade with, and may carry a few small weapons and a simple piece of armor, but nothing fancy, and not at an unlimited amount.
However, a well-renowned general adventurers' shop in a major city will have up to 100,000 Coins to trade with, and will carry a large array of items, ranging from basic rations to enchanted weapons.\\
When a character wants to sell an item, most vendors (that buy items) will pay around 50\% of the item's value, unless they require the item as an ingredient or tool (for example, an alchemist buying ingredients), in which case they will pay close to full price.
Depending on the personality of the vendor, and the general demand and supply in the area, a vendor might pay more, or less.\\
If the vendor is someone who can be bartered with, it is usually a Charisma check against the vendor's willpower.
In general, the GM should approximate the changed price from bartering.\\
However, here's a way to calculate them in order to get a good feeling:\\
The bartering roll can change the price paid (or charged) by between -30\% and +30\%, based on how much above or below the vendor's willpower the roll was.
This range is based on how strong of a barterer the vendor is - the better they are, the larger their range is in the positive, and the smaller their range is in the negative (and vice versa).\\
For example, a bad barterer might pay 30\% more if the bartering check is 5 above their willpower (so 6\% for each point), but pay -30\% if the roll was 30 points below their willpower (so 1\% per point).
A good barterer might only pay 1\% more per point above their willpower, but 10\% less per point below their willpower.
The opposite is true when it comes to the characters buying from the vendor.\\
A person trained in bartering can add their level to their bartering Charisma check, as well as their Willpower when being bartered with.