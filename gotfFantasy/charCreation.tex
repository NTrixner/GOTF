\chapter{Process}\label{ch:charCreationProcess}
Creating a character is a daunting task, but here are some basic steps to go through if you want an interesting character with a bit of background.


\begin{enumerate}
	\item Decide on a basic idea that fits with the rest of the group, the story and the role your character should play in the group.

	\item Decide on a species

	\item Determine your character's basic values

	\item Determine your character's zodiac sign and social rank

	\item Use your starting amount of Learning Points to buy Perks, Spells and Equipment

	\item Determine your character's starting pool values
\end{enumerate}


Let's go through the steps one by one.


\section{Decide on a basic idea}\label{sec:charCreationIdea}

Every character fits a theme. 
Do you want a swashbuckling buccaneer, a raging barbarian, a wise old wizard, maybe a young female priest going on a journey to test her faith?

You may get inspired by your favorite film or book character, or you may conjure up something completely new.
Of course, characters are far more complex than just a basic idea, but at this stage of the creation process, you should create a rather simple stereotype that you want to follow.
Try to describe your character with two or three words, or a short sentence.

Also try to think of the rest of the group, and the overall theme of the adventure you're participating in.
Playing a choleric barbarian in a game of intrigue and mystery may sound fun, but it will probably get stale pretty fast.
Also, a group of 4 empathic thieves will also be pretty boring.

\section{Origins}\label{sec:charCreationOrigins}

Every player character in the GOTF\_Fantasy belongs to one of the Species (see \lq~\nameref{ch:species} \rq).
These determine bonuses to your attributes, as well as your base pool values. 
In addition, some Species gain resistances or other abilities.

After being born of a certain species, your character grew up in a certain culture - use one that your GM created or find one under \lq~\nameref{ch:cultures}\rq.

Additionally, your character also has some amount of backstory before they have started adventuring - an early trade, a heroic deed the performed, or a connection to a certain PC.
Of course, these backstories are complex and for some Species may span large swaths of time.
However, try to find one of the backgrounds listed in \lq~nameref{ch:backgrounds}\rq, and remember to add its perks and modifiers as well.


\section{Determine your attribute values}\label{sec:determineAttributes}

Now that you have an idea and a species, let's talk numbers.
You will have to assign your 7 basic attributes now.
There is different ways to determine these values, and the GM may choose for the whole group.


\subsection{Point Buy}\label{subsec:pointBuy}

When creating a character, start from a base of 2 for every value.
You can assign 33 points freely to any of the 7 attributes, with a maximum of 12.
Be sure to raise at least all of them to a value that you can live with.
With an intellect value of 2, you're not much smarter than your average wombat.\\
After that, add your Racial bonuses.


\subsection{Random roll}\label{subsec:pointRoll}

Alternatively, you can roll 3D4 for every value, either directly assigning the values or rolling first and assigning them afterwards, and adding the racial bonuses afterwards. 


\subsection{Fixed Values}\label{subsec:pointFixed}

Another alternative is using fixed values.
When using this method, you gain the following values to assign to the attributes before applying racial bonuses: 9, 9, 8, 6, 6, 5, 4.


\section{Buy Equipment and Perks}\label{sec:charCreationPerkAndEquip}

Every character usually starts with a total amount of 1600 Learning Points and 2400 Coins.
You can use these Points to buy your character's starting perks and equipment.\\
Your character can only start with rank 1 in any given perk, but can gain rank 1 even for perks that share their cost in a pool (such as skills).\\
After you have chosen perks, you can convert the rest of your Learning Points to Coins in a 1:1 fashion, and add that money to the remaining starting Coins to buy equipment.\\
These are the things your character has learned and acquired before the adventure starts, so be sure that it fits in your character's background story, his personality, social rank and cultural background.\\

\chapter{Starting at higher levels}\label{ch:createHigherLevelChars}
For one reason or another - like a character dying, a player wanting to try out a different character, a new player joining an existing group, or the group starting at a higher level - a player might be able to create a new character at a higher level.
In such cases, the GM decides the starting levels of the character, as well as how many LP and Coins they can spend at character creation, and up to what rank of perks the character can start with.
As a general guideline however, the following idea is suggested:
\begin{itemize}
	\item Start at the minimum XP requirement for the lowest level in the target tier.
	If an existing party has a wide range of tiers between the different game modes, use different tiers for each game mode.
	\item Calculate the amount of LP that a character with that amount of XP would have (see table below) - this is the "Gained LP".
	\item Add the starting LP to the Gained LP - this is the total amount of LP that the character can use for character creation.
	\item Multiply the Gained LP by 1.5, and add the amount of starting Coins to that value - this is the amount of Coins that the Character can spend.
	\item The maximum rank of a given perk should be close to the current path to tier 5.
	For example, if a perk has 10 maximum ranks, and the character starts at tier 2, the characters should be able to start with up to rank 4 in that perk.
	This is not a hard rule, and the GM should be involved in the process.
\end{itemize}