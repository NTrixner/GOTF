\section{Pool Values}\label{sec:pools}
For the GOTF\_F, two main pool values are important, with a third and fourth being tracked, but really being just important for magically and priestly trained characters respectively.
The base number of these values is determined by a character's species and attributes.\\

\subsection{Relevant Attributes}\label{subsec:relevantAttributesPools}
The four Pool Values are each dependent on an Attribute.
Health depends on Vitality, Stamina depends on Strength, Mana depends on Intellect, and Faith depends on Charisma.
This means that each of the Pool Values is equal to a number stemming from a species, plus the relevant attribute.
For example, a Human starts with 25 Health.
With 9 Vitality, a Human would start the game with 34 Health.
If that character then increases their Vitality by 1, their maximum Health would also increase.
Additionally, with each rank in ~\nameref{perk:increasehealth}, Vitality is added to the character's maximum health again.
This also means that a character with 3 ranks in Increase Health would gain 4 additional maximum Health if they increase their Vitality by 1.

\subsection{Health}\label{pool:health} describes a character's ability to withstand any type of damage.
Health can be seen as a combination of resilience, bodily health and luck.
If a character reaches 0 Health, they are \textbf{dead}.
A character with less than one fourth of their Health becomes ~"\nameref{condition:wounded}".
A character with less than one tenth of their Health becomes ~"\nameref{condition:heavilyWounded}".
Health can be restored by resting (See Recovering Pool values, below), by alchemy and by magic.
A character that is dead can not be healed.

\subsection{Stamina}\label{pool:stamina} describes a character's ability to act out tasks that are bodily challenging.
When a character climbs or swims fast, or if a character uses special maneuvers in combat, they lose stamina.
Stamina can be restored by resting (See below).
It usually doesn't take much longer than an hour to completely restore Stamina, so outside a stressful situation, a character usually has full Stamina.

\subsection{Mana}\label{pool:mana} is usually tracked for every character, but only important for magic users.
It describes a character's ability to cast spells.
Casting a spell costs Mana, and a character replenishes mana by eating.

\subsection{Recovering Pool Values}\label{subsec:recoveringPoolValues}
For each of the 4 different Pool values, recovery works the same way, but is triggered by different conditions.
Whenever one of these conditions occur, the characters regain points in that pool value as mentioned below

\subsection{Recovering Health}\label{subsec:recoverHealth}
Aside from using special tools or magic for healing, each character has the ability to recover health in a natural way.
A character's natural healing ability allows them to heal 1 point of health, per 1 hour.
While resting, this healing effect is doubled.
A priest or healer may further improve someone's natural healing ability.

\subsection{Recovering Stamina}\label{subsec:recoverStamina}
Stamina can be recovered by taking a break, regaining breath, not moving too much around, sitting down for a minute or two or drinking a bit of water.
For every 2 seconds a character rests like that (or every AP in combat they spend just resting), they gain back 1 Stamina.

\subsection{Recovering Mana}\label{subsec:recoverMana}
Mana is recovered over time, but the rate at which it recovers can be influenced by the food that the character eats.
A meal usually lasts for 8 hours.
A usual ration provides a recovery of 1d6 points per hour.

\subsection{Recovering Faith}\label{subsec:recoverFaith}
Faith is fully recovered at a certain time of day, after the character holds their daily prayer (which usually takes about 15 Minutes).
The time at which this prayer has to occur is based on the religion of the character, but happens usually at one of the four cardinal times: Dawn, Noon, Dusk, or Midnight.

\subsection{Temporary Pool Values}\label{subsec:temporaryPoolValues}
When a character gains temporary bonuses to their pool values, they function as temporary increase of their maximum values.\\
For example, a character that has 20 out of 30 Health, and gains 10 temporary health, has 30 out of 40 Health.
After the effect ends, they would lose those temporary Health, and their current Health would also be reduced by the same amount.
If that character had taken more than 10 damage in the meantime, this would mean that they would succumb to their wounds once the temporary Health run out.\\