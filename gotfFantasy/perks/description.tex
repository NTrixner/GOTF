\chapter{Overview}\label{ch:perksOverview}
\section{Description}\label{sec:perkDescription}
Perks are moves, abilities and skills a character has learned, either in their past or during their adventures during play.
Learning a perk requires Learning Points (LP) and training time.
Some perks also require a resource to study, like a book, scroll or a trainer.
Perks usually look like this:

\textbf{Name}\\
Perks that have multiple ranks are usually followed by roman numerals detailing their rank.
If such a perk is noted without a rank, the name refers to the first rank of the perk.\\

\textbf{Cost}\\
The cost of the Perk in Learning Points.
A character trying to learn it has to expend this amount of Learning Points to learn it, and they have to take a certain amount of time (in accordance to "Learning Perks" below) based on this value. A perk with a cost of "0" can be taken freely, and often times either has some kind of downside or is one of a set of perks, where only one can be taken. A perk with a cost of "-" is one of the "basic starting perks", and every character is considered to have this perk, at least its first rank if it has ranks.

\textbf{Requirements}\\
A certain set of conditions that have to be met in order to be able to learn this perk.
A character has to fulfil these requirements to be able to learn the perk.

\textbf{Tags}\\
Most perks have one of the following tags.\\
\begin{itemize}
	\item \textbf{Spell, Maneuver, Prayer, Skill etc.} are types of perks.
	 Whenever a rule mentions one of these, all perks with that type are affected by it.\\
	\item \textbf{Active/Passive} describes if a perk is usable or not.
	An active perk is some kind of move or spell, while a passive perk is active all the time.\\
	\item \textbf{Repeatable} perks have multiple ranks, and contain some kind of rank progression which is described after the perk's main description.\\
	\item \textbf{Source} means that a perk needs some type of source, like a trainer, a scroll, an ancient tablet in a long-forgotten language or a book.
	The typical gold value of such a source is the same as the cost of LP of a given rank, in Gold.\\
	A character might be able to achieve a better price from a certain trainer, or by finding a scroll or ancient tablet, but the value of such a service or item is the same as the LP of that Perk Rank.\\
	\item \textbf{Weapon} perks are active perks that have a form of attack roll, followed by a damage roll.
	They are therefore treated as if they were attacks themselves.
	If a different perk changes an attack roll or is based upon it (like Aimed Attack, for example) this perk fulfills the requirement.\\
	\item \textbf{Memory} perks are perks that require some type of ongoing upkeep by the performer (Oftentimes spells).\\
	A creature can only concentrate on an amount of Memory perks equals to their amount of Memory slot.
	A creature's natural amount of memory slots is equals to one third their IN score.
	When a creature's IN changes by some manner, their amount of natural memory slots also changes.
	When a creature's amount of active Memory Perks exceeds the amount of Memory slots (either by taking memory damage, reduced IN, or additional Memory effects becoming active), they have to immediately end the effects of enough Memory Perks of their choice until they fit again.
\end{itemize}

The perk is then usually described in detail.

\section{Learning Perks}\label{sec:learningPerks}
Learning a new perk requires Learning Points and time.
Some perks also require a source, like a scroll, a book or a trainer.
Any character that knows a perk is qualified to be a trainer for it, and can teach other characters and NPCs.

The amount of time required to learn a new perk is based on the perk's Character Point cost and the character's intellect.
It takes at least one day to train a perk, but it is possible to learn multiple perks per day.
One day here is considered to consist of 16 hours of learning with frequent rests.\\

\rowcolors{2}{lightgray}{white}
\begin{tabular}{l | l | l}
	Character Intellect & LP cost covered per day & LP cost covered per hour\\
	1--3 & 100 & 6,25\\
	4--6 & 200 & 12,50\\
	7--9 & 300 & 18,75\\
	10--12 & 400 & 25\\
	13+ & 500 & 31,25
\end{tabular}