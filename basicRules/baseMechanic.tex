\section{The Base mechanic}\label{sec:basemechanic}

\subsection{The Meta Mechanic}\label{subsec:metamechanic}
At the beginning of a scene, the GM describes the scene, as well as any points of interest.
The Players then decide what their player characters want to do.
These actions are then adjudicated.
After actions are adjudicated, the scene has changed and the GM describes the changed scene, or the next scene.

\subsubsection{Action Adjudication}\label{subsubsec:actionadjudication}
Whenever an action happens, it is adjudicated.
The GM decides two things: A) Can the action succeed.
If the action is impossible to succeed, the GM should usually tell the Player.
B) Can the action fail.
If an action can not fail, there is no need to make a roll.\\
If an action is both possible to fail as well as possible to succeed, a character must make an appropriate check.
Most of these checks are Attribute Checks, as described in the next section.\\


\subsection{Attribute Checks}\label{subsec:attributechecks}
Every check is solved by a roll of a D20 You add a specific modifier to your roll, and maybe get additional bonuses if you're trained in that specific task.
Whenever you roll a check, you compare the outcome of your roll to a Threshold (Th).
If your roll beats that Th, you are able to perform that task.
If your value is lower than the Th, you can not perform the task.
If your roll value and the Th are equal, you roll again.

\subsubsection{Contests}\label{subsubsec:contests}
Sometimes, two creatures compete in a task.
In order to get the outcome, both people roll a D20 and add their relative modifiers, and optional additional dice.
The one with the higher result wins the contest.
If both rolls are the same, they are rolled again until they are not the same anymore.

\subsubsection{Helping Others and working together}\label{subsubsec:helping}
Sometimes, a character may be able to help another character with a difficult task.
In such a case, the player describes how exactly they are helping, and the GM adjudicates if it's one of three general scenarios.\\
\textbf{Guided acting} occurs if one character tells another character what to do in order to achieve a task.
In this case, the acting character may add a bonus set by the System to the roll for every rank of Proficiency Perk that the guiding character has for said task.\\
\textbf{Direct Help} If a character helps another character directly in a certain task, the helping character may have to roll for the specific way in which they are helping.
If they succeed, the Threshold of the original task is reduced.\\
\textbf{Working together} If two or more characters are working together, every character makes a roll.
Depending on the nature of the task, either the group succeeds if one of them succeeds, or the group fails if one of them fails.\\
Depending on the task, working together might change the Th of the given task.\\

\subsubsection{Retries}\label{subsubsec:retries}
Rolls determine ability, not (just) luck.
his means, that unless the circumstances change, the outcome of the roll is fixed.
Retries are not an option.
If the circumstances change, rolls can be retried.
This can mean stress, new knowledge, new abilities or a changed environment.


\subsection{Rounding}\label{subsec:rounding}
Every time a division happens and the result would be a fraction when an integer is needed, you should round up.

\subsection{Modifiers Types}\label{subsec:modifiertypes}
When modifiers are applied to bonuses and maluses, they usually have a type associated with them.
In general, a creature only profits the strongest bonus of a given type, and only suffers the strongest malus of a given type, but different types of bonuses and maluses are summed up.
Additionally, some bonuses and maluses apply dice instead of flat values.\\
In such cases, the largest amounts of the same sized dice (both positive and negative) apply to the checks.
Additionally, positive and negative dice of equal size cancel each other out (for example, a creature with +1d4 and -1d4 has a bonus of 0) before the actual roll.\\
Lastly, some modifiers are multiplicative.
Multiple multiplicative modifiers of different types are treated as if they were themselves summations, meaning the total of a xN and a xM multiplier is x(N+M-1) - basically, a multiplicative Modifier is a \lq +X\% \rq bonus.
Multipliers are applied first, fractions second, and then dice bonuses/maluses and flat bonuses/maluses.\\

\subsection{Passage of Time}\label{subsec:passageoftime}
Actions usually take at least some time.
A GM will adjudicate an action at the moment that it starts, and then time will pass while they are acting.\\
The amount of time passing should be clear to the players, and the GM should always be aware what character is performing which action.\\
The amount of time passing during an action is determined by the action itself, and varies during different frames of gameplay.
For example, overland travel might be tracked in hours or even days, while searching a room might take minutes.
All the while, combat will only take a few minutes, maybe even just seconds.

\subsubsection{Action Points}\label{subsubsec:actionpoints}
Action Points are a special unit used within the rest of the document.
Action Points (\textbf{AP}) are used to measure time in action-heavy scenes.
The real duration of an Action Point is determined by the system.