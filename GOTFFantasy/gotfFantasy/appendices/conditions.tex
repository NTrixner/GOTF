\chapter{Conditions}\label{ch:conditions}
\section{Wounded}\label{condition:wounded}
A character that falls below one fourth of their health becomes wounded.
While wounded, a character suffers a -2 Situation Malus on all weapon attack rolls, strength checks and vitality checks.

\section{Heavily Wounded}\label{condition:heavilyWounded}
A character that falls below one tenth of their health becomes heavily wounded.
While heavily wounded, a character suffers a -5 Situation Malus on all weapon attack rolls, strength checks and vitality checks.

\section{Exhausted}\label{condition:exhausted}
While exhausted, a character suffers a -1d4 Exhaustion Malus on all checks.

\section{Heavily Exhausted}\label{condition:heavilyExhausted}
While heavily exhausted, a character suffers a -1d6 Exhaustion Malus on all checks.

\section{Entranced}\label{condition:entranced}
An entranced character is only able to perceive the source of their entrancement.
Any other Perception checks automatically fail.
Taking damage ends this condition.

\section{Frightened}\label{condition:frightened}
A frightened character can not move freely towards the cause of their fear.
If the effect has no specific cause, they can only cower.
While frightened, a character suffers from a -5 Fear Malus on attack rolls.
If the source of a character's fear is removed, this effect ends.

\section{Helpless}\label{condition:helpless}
A helpless creature is unable to react to attacks.
As a result, melee attacks against a helpless creature automatically hit, and count as critical hits.

\section{Hastened}\label{condition:hastened}
A hastened character can act twice as fast as normal.
This means that any action's AP is reduced by half, to a minimum of 1.
Hastened and Hindered cancel each other out completely.
A hastened character who is affected by the Hindered effect loses both conditions.

\section{Hindered}\label{condition:hindered}
A hindered character can act half as fast as normal.
This means that any action's AP is doubled.
Hastened and Hindered cancel each other out completely.
A hindered character who is affected by the Hastened effect loses both conditions.

\section{Invisible}\label{condition:invisible}
An invisible character can not be seen.
When an invisible character tries to move stealthily, they can roll twice, taking the higher result.
Any attack rolls against invisible suffer a -20 Situation Malus.

\section{Prone}\label{condition:prone}
A prone character is crawling or lying on the ground.
Melee attack rolls against prone characters have a Situation Bonus of +5, ranged attack rolls against prone characters have a Situation Malus of -5.
While being prone, moving costs twice the AP for a character.
Standing up is a form of movement, and takes 2 AP if the character is unarmored or in light armor, 4 AP if the character is in medium armor and 6 AP if they are in heavy armor.

\section{Grabbed}\label{condition:grabbed}
A grabbed creature is being held by some other creature or effect.
As a result, the creature suffers a -10 Situation Malus to attacks against anyone that isn't the creature or effect grabbing them.
Additionally, attacks against a grabbed creature gain a +10 Situation Bonus.

\section{Pinned}\label{condition:pinned}
A pinned creature is being held down by a grabber.
They are considered restrained, prone and helpless.

\section{Restrained}\label{condition:restrained}
A restrained character can not move.

\section{Sleeping}\label{condition:sleeping}
A sleeping character is prone and unconscious.
Loud noises, being moved or taking damage end this condition immediately.
However, a character can make a vitality check to force themselves to sleep in harsh conditions.

\section{Paralyzed}\label{condition:paralyzed}
A paralyzed character can not act and is considered helpless and restrained.
If the character was paralyzed in a non-stable stance, they also fall prone upon becoming paralyzed.

\section{Unconscious}\label{condition:unconscious}
An unconscious character is helpless.
An unconscious character can not move or act and suffers a -20 Situation Malus to perception checks.